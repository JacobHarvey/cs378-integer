\hypertarget{index_intro_sec}{}\section{Introduction}\label{index_intro_sec}
\char`\"{}\+Background\+:
\+Project managers, such as the U\+N\+I\+X utility make, are used to maintain large software projects made up from many components. Users write a project file specifying which components (called tasks) depend on others and the project manager can automatically update the components in the correct order. .\char`\"{} \href{http://www.spoj.com/problems/PROBTNPO/}{\tt http\+://www.\+spoj.\+com/problems/\+P\+R\+O\+B\+T\+N\+P\+O/} \hypertarget{index_time_est}{}\section{Time}\label{index_time_est}
Estimate\+: 8 hours Actual\+: 10 hours\hypertarget{index_install_sec}{}\section{The Problem}\label{index_install_sec}
Write a program that reads a project file and outputs the order in which the tasks should be performed Input For simplicity we represent each task by an integer number from {\bfseries 1,2,...,N} (where {\bfseries N} is the total number of tasks). The first line of input specifies the number {\bfseries N} of tasks and the number  of rules, such that  leq 100,; Mleq 100. The rest of the input consists of  rules, one in each line, specifying dependencies using the following syntax\+: {\bfseries  T\+\_\+0 k T\+\_\+1 T\+\_\+2 ... T\+\_\+k} This rule means that task number  depends on  tasks {\bfseries T\+\_\+1, T\+\_\+2, ... T\+\_\+k} (we say that task  is the target and {\bfseries T\+\_\+1 ... T\+\_\+k$<${\bfseries } $>$ are dependents). Note that tasks numbers are separated by single spaces and that rules end with a newline. Rules can appear in any order, but each task can appear as target only once. Your program can assume that there are no circular dependencies in the rules, i.\+e. no task depends directly or indirectly on itself. Output The output should be a single line with the permutation of the tasks  N{\bfseries to} be performed, ordered by dependencies (i.\+e. no task should appear before others that it depends on). To avoid ambiguity in the output, tasks that do not depend on each other should be ordered by their number (lower numbers first). }